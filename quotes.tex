\documentclass[a4paper]{article}

\usepackage{amogus}
\usepackage{multicol}
\usepackage[osf]{libertinus}
\usepackage[xspace]{ellipsis}
\geometry{margin = 1cm}
\pagestyle{plain}

\newcommand{\heading}[1]{\textbf{#1}}
\newcommand{\linguist}[1]{\textemdash~\textsc{#1}}

\begin{document}
\begin{center}
  \Large \bfseries \scshape \textls[25]{english language linguist quotes} \normalsize \normalfont \makebox[0pt][l]{(last updated \today)}
\end{center}
\begin{multicols}{2}
  \heading{General}
  \begin{itemize}
    \item Use what language you will, you can never say anything but what you are. \linguist{Ralph Waldo Emerson} (American essayist, lecturer, and poet)
    \item Like everything metaphysical the harmony between thought and reality is to be found in the grammar of the language. \linguist{Ludwig Wittgenstein} (Austrian-British philosopher)
    \item Male supremacy is fused into the language, so that every sentence both heralds and affirms it. \linguist{Andrea Dworkin} (American radical feminist and writer)
    \item Language most shows a man, speak that I may see thee. \linguist{Ben Jonson} (English playwright, poet, actor, and literary critic of the 17th century)
    \item Language is the dress of thought. \linguist{Samuel Johnson} (English essayist, moralist, literary critic, biographer, editor, and lexicographer)
    \item Language needs the chance to constantly renew itself. \linguist{Gunter Grass} (German novelist, poet, playwright, illustrator, graphic artist, sculptor, and recipient of the 1999 Nobel Prize in Literature)
    \item A lot of friendships and connections absolutely depend upon a sort of shared language, or slang. Not necessarily designed to exclude others, this can establish a certain comity and, even after a long absence, re-establish it in a second. \linguist{Christopher Hitchens} (English-American author, columnist, essayist, orator, religious and literary critic, social critic, and journalist)
    \item Political correctness does not legislate tolerance; it only organises hatred. \linguist{Jacques Barzun} (American historian)
    \item Euphemism is a human device to conceal the horrors of reality. \linguist{Paul Johnson} (English journalist, historian, speechwriter, and author)
    \item Euphemism is a euphemism for lying. \linguist{Bobbie Gentry} (American singer-songwriter)
    \item All Australian accents have changed, but they change through the speech of young people. Once you reach your 20s, your accent doesn't change much. \linguist{Felicity Cox} (Australian phonetician)
    \item Tact is kind; diplomacy is useful; euphemism is harmless and sometimes entertaining. \linguist{Julian Burnside} (Australian barrister, human rights and refugee advocate, and author)
    \item Swearing is more than a way of expressing extreme, usually negative emotion. It's a social adhesive that units and binds us more than any other \dots it opens people up, cements relationships. \linguist{Michael Stuchbery} (Australian author)
    \item Slang works much like masonic mortar to stick members of a group together --- and of course at the same time to erect barriers between them and the outside. \linguist{Kate Burridge} (Australian linguist)
    \item `Bloody' has now become an important indicator of Australianness and of cultural values such as friendliness, informality, laid-backness, mateship --- and perhaps even the Australian dislike and distrust of verbal and intellectual graces. \linguist{Kate Burridge}
    \item If you lose your language, you lose your personality, your character and who you are. \linguist{Hugh Lunn} (Australian journalist and author)
    \item The expression of language has become richer because of the internet. \linguist{David Crystal} (English linguist)
    \item Our (Australian) accent is a product of our social history. \linguist{Felicity Cox}
    \item Teenagers use language as a kind of identity badge that has the effect of excluding adults. \linguist{Pam Peters} (Australian linguist)
    \item Australian English is becoming well known for its quirky, larrikin, idiosyncratic creativeness. \linguist{Roland Sussex} (Australian linguist)
    \item Texting has added a new dimension to language use, but its long-term impact is negligible. It is not a disaster. \linguist{David Crystal}
  \end{itemize}
  \heading{Prescriptivism vs Descriptivism}
  \begin{itemize}
    \item All languages meet the social and psychological needs of their speakers, are equally deserving of scientific study, and can provide us with valuable information about human nature and society. \linguist{Crystal}
    \item Prescriptivism is often based on ``religious and philosophical preconceptions.'' \linguist{Jen Aitchison}
    \item Language is constantly evolving and this is part of the evolutionary process. \linguist{Bruce Moore}
  \end{itemize}
  \heading{Standard English and Text Speak}
  \begin{itemize}
    \item The vast majority of spelling rules in English are irrelevant. They don't stop you understanding the word in question. \linguist{Crystal}
    \item Spelling was only standardised in the 18th century. In Shakespeare's time you could spell more or less as you liked. \linguist{Crystal}
    \item Standard English spelling is an absolute criterion of an educated background. \linguist{Crystal}
    \item Sounds are too volatile and subtle of legal restraint. \linguist{Samuel Johnson} (Author of the first English dictionary)
    \item \dots tendency to believe the notion that face can be saved by following the practices recommended by the grammarian. \linguist{Crystal}
  \end{itemize}
  \heading{Language and Identity}
  \begin{itemize}
    \item More than anything else, language shows we belong, providing the most natural badge or symbol or public and private identity. \linguist{Crystal}
    \item All subsystems of language can have an influence on how we mark identity through language. \linguist{Macmillan textbook}
    \item Ethnicity is an important part of social identity and something that people want to demonstrate through their use of language. \linguist{Burridge and Mulder}
    \item A broad Australian accent and the use of conventionally tabooed language become desirable macho markers of gender identity. \linguist{Burridge and Mulder}
  \end{itemize}
  \columnbreak
  \heading{Australian English}
  \begin{itemize}
    \item Mark out a community as different from others in history, its way of life, its attitudes and its traditions. \linguist{Macquarie Dictionary Website}
    \item Australian English can be seen as the natural development of a post-imperialist colony, through divergent linguistic development. \linguist{Mitchell and Delbridge}
    \item In periods of patriotism, it was felt that ``swearing and a strong, broad Australian accent,'' for example, are associated with toughness and strength and these can be highly valuable qualities. \linguist{Burridge}
    \item Cultivated Australian English can be seen as snobbish and one often encounters hostile or amused reactions to the cultivated accent. \linguist{Burridge}
    \item Australian English functions as a significant and extremely powerful symbol of national identity. \linguist{Macquarie University}
  \end{itemize}
  \heading{Slang}
  \begin{itemize}
    \item Slang is language of a highly colloquial and contemporary type. \linguist{Burridge and Allan}
    \item The use of slang is a means of marking social or linguistic identity. \linguist{Crystal}
    \item Swearing can become a dominant linguistic trait. \linguist{Crystal}
  \end{itemize}
  \heading{Jargon}
  \begin{itemize}
    \item A variety of language used among people who have a common work-related or recreational interest. \linguist{Burridge and Allan}
    \item Chief linguistic element that shows social togetherness. \linguist{Crystal}
    \item Unless you are a member of a clique \dots it's gibberish. \linguist{Steve Pinker}
    \item One person's jargon is another person's vocabulary. \linguist{Ilana Mushin}
  \end{itemize}
  \heading{Euphemism}
  \begin{itemize}
    \item Serve direct human interests by avoiding those things which threaten to  cause offence and distress. \linguist{Burridge}
    \item You could think of euphemism as a kind of linguistic dressing. It can be decorative, flavour enhancing, concealing \dots \linguist{Burridge}
    \item Latin words sound scientific and therefore appear to be technical and clean whereas their Anglo-Saxon counterparts are taboo. \linguist{Fromkin, Blair, and Collins}
    \item Euphemism treadmill \dots the new word becomes tainted, prompting the search for yet another fresh word. \linguist{Steve Pinker}
    \item Euphemisms are certainly motivated by the desire not to be offensive but they are more than just linguistic fig leaves. \linguist{Burridge}
  \end{itemize}
  \heading{Dysphemism}
  \begin{itemize}
    \item They remain in the language to vent strong emotion. \linguist{Fromkin, Blair, and Collins}
    \item Swearing has important social function. \linguist{Crystal}
    \item The focus of offensive language has definitely shifted from the religious to the secular, especially to matters relating to sexual and bodily functions. \linguist{Burridge}
    \item Laws against profanity, blasphemy and (sexual) obscenity have been replaced in heinousness by sanctions against -IST language. \linguist{Burridge and Allan}
    \item Words and language are not intrinsically good or bad but reflect individual or society values. \linguist{Fromkin, Blair, and Collins}
    \item Words are often sacrificed when they take on secondary, emotionally charged meanings. \linguist{Pinker}
    \item It is generally accepted that `cunt' is the most tabooed word in the English language. \linguist{Burridge and Allan}
  \end{itemize}
  \heading{Discriminatory Language}
  \begin{itemize}
    \item Women are rendered invisible in the language when the masculine pronoun `he' is used. \linguist{Fromkin, Blair, and Collins}
    \item There are even legally recognised sanctions against what broadly might be called IST-language. \linguist{Burridge}
    \item The whole framework \dots so deep rooted that it goes unnoticed. \linguist{Crystal.}
  \end{itemize}
  \heading{Political Correctness}
  \begin{itemize}
    \item Political Correctness brought a fresh awareness of the nature of regional and ethnic identity, which led to greater valuing of linguistic diversity. \linguist{Crystal}
    \item PC language deliberately throws down the gauntlet and challenges us to go beyond the content of the message and acknowledge the assumptions on which our language is operating. \linguist{Burridge and Allan}
    \item The suggestion that by eradicating offensive language we would eradicate social attitudes and inequalities betrays a lack of understanding of how language works. \linguist{Crystal}
    \item A healthy expansion of moral concern. \linguist{Noam Chomsky}
  \end{itemize}
  \heading{Political Language}
  \begin{itemize}
    \item The whole tendency of modern prose is away from concreteness. \linguist{Orwell}
    \item Language that makes the bad seem good, the negative seem positive and the unacceptable appear attractive. \linguist{Crystal}
    \item The truth is less significant than the political conquest. \linguist{Watson}
    \item In our time, political speech and writing have largely been the defence of the indefensible. \linguist{Orwell}
    \item Political language is designed to make lies sound truthful, murder respectable, and to give the appearance of solidarity to pure wind. \linguist{Orwell}
    \item Language has been made the machine of business and politics. \linguist{Watson}
      \item Designed to intimidate the populous through mystification. \linguist{Thorne}
      \item Truth is the first casualty of war. \linguist{US Senator Johnson, 1918}
      \item It is language which pretends to communicate but really doesn't. \linguist{Lutz}
  \end{itemize}
  \heading{Polite Language}
  \begin{itemize}
    \item What counts as polite behaviour varies between human groups. \linguist{Burridge and Allan}
    \item Negative politeness avoids intruding and so emphasises social distance. \linguist{Holmes}
    \item Different cultures and linguistic groups express politeness differently. \linguist{Holmes}
  \end{itemize}
\end{multicols}



\end{document}
